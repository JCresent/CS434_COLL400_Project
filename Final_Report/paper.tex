% TODO fill in your paper title
\newcommand{\PaperTitle}{A Comparative Analysis of Machine Learning Algorithms for Website Traffic Classification from Network Packets }
% TODO fill in your paper number when you get it
\newcommand{\PaperNumber}{XXX}

\documentclass[10pt,sigconf,letterpaper,nonacm]{acmart}

%%%%%%%%%%%%%%%%%%%%%%%%%%%%%%%%%%%%%%%%%%%%%%%%%%%%%%%%%%%%%%%%%%%%%%%%%%%%
% This is the preamble; include packages as you see fit.
% Here are a few recommendations:
% \usepackage{color}
% \usepackage{graphicx}
% \usepackage[labelformat=simple]{subcaption}
% \usepackage{xspace}
% \usepackage{multirow}
% \usepackage[ruled,vlined]{algorithm2e}
% \usepackage{ulem}
% \normalem

%%%%%%%%%%%%%%%%%%%%%%%%%%%%%%%%%%%%%%%%%%%%%%%%%%%%%%%%%%%%%%%%%%%%%%%%%%%%

\begin{document}

\title{\PaperTitle}

\author{Matthew Berthoud, Lake Bradford, Justin Cresent, Will Katabian}
\affiliation{
  \institution{William \& Mary}
  \city{Williamsburg}
  \state{Virginia}
  \country{USA}
}

\begin{abstract}
Accurately Identifying web traffic destination and origins is crucial for the efficiency of a network. This project explores the potential of machine learning in reference to web traffic classification based on the 
analysis of network packets. 
We monitored and analyzed web traffic data from ChatGPT, Blackboard, and Linkedin, with the objective 
of building models which will be able to predict the web traffic origin of a specific packet. The collection of data was performed
using wireshark, then the data was reformatted to eliminate bias and get more accurate results.
Using the collected data we then trained four models which had varying levels of accuracy, Logistic Regression (56\%), 
K-Nearest Neighbors (77\%), Random Forest (78\%), and finally a neural
network (80\%). This project shows the importance of machine learning within the field of 
network traffic analysis as automaton is much more efficient and precise compared to manually
examining web traffic, especially in a scale as large as the internet. One example of our project's significance is that this can be crucial data analysis 
for network administrators and security professionals, whom would examine the network for malicious traffic.

\end{abstract}

\keywords{Network Traffic, Machine Learning, Web Traffic Classification, Network Packets, Data Analysis, Neural Networks, Random Forest, K-Nearest Neighbors, Logistic Regression}

\maketitle

\section{Introduction}
The ability to monitor and analyze network traffic is crucial for the efficiency and security of a network \cite{10.1145/2388576.2388608}. 
It helps to manage the overall network performance, detect and prevent malicious activities, and ensure the network is operating as intended.
The present day internet is composed of a vast variety of diverse web traffic, of which requires a more sophisticated approach to analyze and classify network traffic.\cite{10.5555/3432601.3432608}
This project investigates a variety of machine learning algorithms to see which most accurately and effectively classify web traffic based on data from a set of captured network packets. 

  Traffic classification is very significant in practice, if done accurately and effectively \cite{10.1109/TNET.2014.2320577}. For reasons mentioned previously, the observed capabilities of this and other projects 
  can be crucial for network administrators and security professionals, whom would examine the network for malicious traffic.

  For this project, many sets of packet data were collected from three popular websites: ChatGPT, Blackboard, and Linkedin. These sets were then merged into a single dataset, which was then split into training and testing sets. 
  Three models were then trained and tested on this data: Logistic Regression, K-Nearest Neighbors, Random Forest, and Neural Network \cite{scikit-learn}. 

  A thorough evaluation of the models was conducted through testing and validation sets, hyperparameter optimization along with cross-validation, and computing accuracy and Macro F1 scores. The results showed that the Neural Network model 
  acheived the highest accuracy of $80\%$ and Macro F1 score of BLANK. %%TODO fill in the score.%%
  These scores show the model's ability to accurately classify the selected websites based on captured network packet data.

  Overall, the significance of this project is seen in its use of Machine Learning and the extensions of these applications into the discipline of network traffic analysis. This practice has potential to be utilized in real-world applications, and many sources
  proving it already is \cite{10.5555/3432601.3432608}. 

\section{Proposed Method}
This section presents the methodology for how this project goes about capturing network packet data, analyzing the data, and classifying it based on its website of origin.

\subsection{Capturing Data}
\subsection{Data Preparation and Processing}
\subsection{Employed models}
\section{Evaluation}
This section comprehensively evaluates our models and their performance on the dataset and validating sets.
\subsection{Dataset}


\section{Discussion \& Future Work}

\section{Conclusion}

% Note from the CFP that this section must include a statement about
% ethical issues; papers that do not include such a statement may be
% rejected.

%%%%%%%%%%%%%%%%%%%%%%%%%%%%%%%%%%%%%%%%%%%%%%%%%%%%%%%%%%%%%%%%%%%%%%%%%%%%
% We're in the endgame now

\bibliographystyle{ACM-Reference-Format}
\bibliography{refs}
\cite{10.1145/2388576.2388608}
\cite{10.5555/3432601.3432608}
\cite{10.1109/TNET.2014.2320577}
\cite{scikit-learn}

\end{document}
