% TODO fill in your paper title
\newcommand{\PaperTitle}{A Comparative Analysis of Machine Learning Algorithms for Website Traffic Classification from Network Packets }
% TODO fill in your paper number when you get it
\newcommand{\PaperNumber}{XXX}

\documentclass[10pt,sigconf,letterpaper,nonacm]{acmart}

%%%%%%%%%%%%%%%%%%%%%%%%%%%%%%%%%%%%%%%%%%%%%%%%%%%%%%%%%%%%%%%%%%%%%%%%%%%%
% This is the preamble; include packages as you see fit.
% Here are a few recommendations:
% \usepackage{color}
% \usepackage{graphicx}
% \usepackage[labelformat=simple]{subcaption}
% \usepackage{xspace}
% \usepackage{multirow}
% \usepackage[ruled,vlined]{algorithm2e}
% \usepackage{ulem}
% \normalem

%%%%%%%%%%%%%%%%%%%%%%%%%%%%%%%%%%%%%%%%%%%%%%%%%%%%%%%%%%%%%%%%%%%%%%%%%%%%

\begin{document}

\title{\PaperTitle}

\author{Matthew Berthoud, Lake Bradford, Justin Cresent, Will Katabian}
\affiliation{
  \institution{William \& Mary}
  \city{Williamsburg}
  \state{Virginia}
  \country{USA}
}

\begin{abstract}
Accurately Identifying web traffic destination and origins is crucial for the efficiency of a network. This project explores the potential of machine learning in reference to web traffic classification based on the 
analysis of network packets. 
We monitored and analyzed web traffic data from ChatGPT, Blackboard, and Linkedin, with the objective 
of building models which will be able to predict the web traffic origin of a specific packet. The collection of data was performed
using wireshark, then the data was reformatted to eliminate bias and get more accurate results.
Using the collected data we then trained four models which had varying levels of accuracy, Logistic Regression (56\%), 
K-Nearest Neighbors (77\%), Random Forest (78\%), and finally a neural
network (80\%). This project shows the importance of machine learning within the field of 
network traffic analysis as automaton is much more efficient and precise compared to manually
examining web traffic, especially in a scale as large as the internet. One example of our project's significance is that this can be crucial data analysis 
for network administrators and security professionals, whom would examine the network for malicious traffic.

\end{abstract}

\keywords{Network Traffic, Machine Learning, Web Traffic Classification, Network Packets, Data Analysis, Neural Networks, Random Forest, K-Nearest Neighbors, Logistic Regression}

\maketitle

\section{Introduction}

\section{Proposed Method}

\section{Evaluation}

\section{Discussion \& Future Work}

\section{Conclusion}

% Note from the CFP that this section must include a statement about
% ethical issues; papers that do not include such a statement may be
% rejected.

%%%%%%%%%%%%%%%%%%%%%%%%%%%%%%%%%%%%%%%%%%%%%%%%%%%%%%%%%%%%%%%%%%%%%%%%%%%%
% We're in the endgame now

\bibliographystyle{ACM-Reference-Format}
\bibliography{refs}

\end{document}
